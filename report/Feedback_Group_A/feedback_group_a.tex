\documentclass[a4paper,11pt]{article}
\usepackage{amsmath}
\usepackage{cancel}
\usepackage{cases}
\usepackage{graphicx}
\usepackage{listings}
\usepackage{color}
\usepackage{float}
\usepackage{amssymb}
\usepackage{bbm}
\usepackage{subcaption}
\usepackage{csquotes}
\usepackage{physics}    % For d/dx symbols
\usepackage[toc,page]{appendix}
\usepackage[backend=biber,style=authoryear]{biblatex}
% \bibliographystyle{plain.bst}
\bibliography{references}

\title{Feedback for Group A - from Group H \\ Course: FRTN70 - Project in Systems Control and Learning}
\date{}

\setlength{\parindent}{0cm} % Determines the indentation for new paragraphs
\setlength{\parskip}{1em}

\begin{document}

\maketitle

Very interesting project and well written report.
It will be exciting to see the final result.
The problem and the solution are both explained in a way that is understandable for someone with limited background in this field.
Good job on that!

It is very interesting that you used different models to solve the same problem.
We are curious to see how they all perform.

You managed to fill out a great discussion with the limited results you have so far.
Really good with several suggestions on how to improve the models in future work.

We have some suggestions for what can be analyzed and added to the report as well as some minor pointers on typos.

It would be nice to see a graph showing the how the total investment changes over the test period for all (or most) models.
A visual way to interpret the Sharpe ratio to get an idea of what it means in practice to follow and use the different models.

What does the initial investment look like at $t = 0$?
Is the investment equally distributed in all assets in the portfolio in the beginning, or perhaps weighted for the values of the stocks?

Some pointers presented below.

\begin{itemize}
	\item Typo in the word 'be/bee'. The $\beta_2$ parameter needs to \textbf{bee} so that $\beta_2 \in (0, 1)$ and $\mu_2$ can be calculated as:
	
	\item Sentence at the end of section "3. Implementation" is incomplete.
	
	\item Table numbering is a little confusing. Table-3 comes up before Table-2.
	
	\item In various places the 'e' is missing in Sharpe. 
	\begin{itemize}
		\item Table-2 caption.
		
		\item Table-2 in the label.
		
		\item Section peformance metrics: To evaluate the forecasting methods used, two different performance metrics are studied, the first being the Sharp Ratio.
		
		\item By calculating the Sharp Ratio the investments returns is compared to it’s risk.	
	\end{itemize}
	
	\item In section 2.1 the risk aversion parameter is mentioned as aviation parameter.


	\item For equations (11) and (14) it would look nicer with the word "if" in regular text rather than math mode. \verb'\text{if}' can be used in math mode to achieve this (this is a completely subjective suggestion).
	
	\item In section 2.1, while explaining the methods that can be used to calculate the covariance, the Ledoit and Wolf’s method is mentioned but not referenced.
	
	\item Typo in table-1 $T[SCM$. Should be an underscore.
\end{itemize}


Table-3 is difficult to read.
Perhaps the table can be modified to have one hyperparameter per line in the final column, see example below.
\url{https://www.tablesgenerator.com/latex_tables#}
Can be used to generate latex code for tables.

\begin{table}[hb]
\begin{tabular}{|l|l|l|}
\hline
Return Forecasting method & Covariance Forcasting method & Optimal hyper parmaeters                                                                 \\ \hline
SMA                       & SCM                          & \begin{tabular}[c]{@{}l@{}}$\gamma_t = $,\\ $T_{SMA} = $,\\ $T_{SCM} = $\end{tabular}    \\ \hline
SMA                       & EMACM                        & \begin{tabular}[c]{@{}l@{}}$\gamma_t = $,\\ $T_{SMA} = $,\\ $T_{LW+SCM} = $\end{tabular} \\ \hline
\dots      & \dots         & \dots                                                                     \\ \hline
\end{tabular}
\end{table}

\end{document}